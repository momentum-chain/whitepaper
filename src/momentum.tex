\documentclass[conference]{IEEEtran}
\IEEEoverridecommandlockouts
% The preceding line is only needed to identify funding in the first footnote. If that is unneeded, please comment it out.
\usepackage{cite}
\usepackage{amsmath,amssymb,amsfonts}
\usepackage{algorithmic}
\usepackage{graphicx}
\usepackage{textcomp}
\usepackage{xcolor}
\def\BibTeX{{\rm B\kern-.05em{\sc i\kern-.025em b}\kern-.08em
    T\kern-.1667em\lower.7ex\hbox{E}\kern-.125emX}}
\begin{document}

\title{Momentum\\
{\footnotesize \textsuperscript{A general purpose blockchain platform utilizing proof-of-polularity}}
}

\author{\IEEEauthorblockN{1\textsuperscript{st} Given Name Surname}
\IEEEauthorblockA{\textit{dept. name of organization (of Aff.)} \\
\textit{name of organization (of Aff.)}\\
City, Country \\
buhrmi@gmail.com}
}

\maketitle

\begin{abstract}
This document outlines the key concepts and operational mechanics of a proposed blockchain
system utilizing a novel consensus algorithm called proof-of-popularity.
\end{abstract}

\begin{IEEEkeywords}
decentralized systems, consensus, blockchain, smart contracts, cryptocurrency
\end{IEEEkeywords}

\section{Introduction}
Momentum is an implementation of a blockchain system utilizing the proof-of-popularity mechanism.

Most conventional blockchain systems rely on a consensus mechanism which is either based on proof-of-work
or proof-of-stake. Both come with their own set of problems: Proof-of-work requires a large amount of computational power,
which draws a lot of criticism for its wastefulness in our resource-constrained world. Proof-of-stake
usually involves pre-mining, which assigns a large amount of its value to predetermined wallet addresses.

Proof-of-popularity addresses these problems and offers an alternative to the aforementioned consensus mechanism.

\section{Overview}
In a proof-of-popularity-based blockchain, the valid chain is always the chain that contains the block with the highest popularity.
Nodes work together to determine the popularity of any given block and ensure its integrity and thus the integrity of the network itself.

On a high level, a well-behaved node in the Momentum network acts like this:

\begin{enumerate}
\item Await total popularity advertisements from peers.
\item Download block with highest popularity
\item Validate block
\item Await next timestamp
\item Generate new block candidate
\item Send block hash to peers
\item Await block hashes and signatures
\item Send signature to the producer of the block with the highest $total rank$.
\item Append block candidate to chain and advertise the new popularity
\item Go to 1
\end{enumerate}

\section{Block validation}

A block is valid if:

\begin{itemize}
\item There is no block with the same or lower timestamp which has a higher popularity.
\item The referenced previous block is valid.
\item The timestamp of the block is exactly 1 greater than the timestamp of the referenced previous block.
\item The timestamp of the block is lower than or equal to the current timestamp.
\item The $rank$ value is correct
\item The $total rank$ value equals the $total rank$ value of the referenced previous block plus the $rank$ value of the block.
\end{itemize}

\section{Definitions}

\subsection{Node}

A participating Machine (Computer) in the Momentum Network. 

\subsection{Rank}

For each timestamp there exists a global order of all accounts that hold Momentum. This order is calculated using a randomized
weighting function that takes a timestamp and the set of all accounts as input, and returns an ordered list of accounts.
This order is used to determine the priority for each account to produce the next block. The rank of the block producers is included
in the block header.

Each block has a $rank$ and $total rank$ value within its header. The $rank$ value specifies its rank in the priority list during the timestamp of block production.
The $total rank$ value is the sum of all $rank$ values in the chain.


\subsection{Popularity}

The popularity of a block. This is the sum of the popularity of the previous block plus the weight of all collected signatures included in the block header.
The chain that contains the most popular block is the valid (authorative) chain.


\section{Implementation}


\subsection{Storage}
While it's up to the implementation to decide on its data storage strategy, it makes sense to utilize a distributed hash table and leverage already existing implementations.

\subsection{Peer Discovery}
TODO: Utilize DHT implementation, for example Kademlia

\section{Considerations}

\subsection{Network speed}
The distribution of the last block might take a long time, especially as the network
grows and the number of signatures per block increases. The timestamp interval has to be chosen in a way
to allow for a good amount of time for the last block to propagate through the DHT to many nodes.

\section{Conclusion}
Momentum and its proof-of-popularity-based approach offers many benefits over conventional blockchain systems.
For one, it does not utilize proof-of-work,
and as such fits nicely into a resource-constrained and climate-conscious environment. There also is no pre-mining of
any form, which is usually required in a proof-of-stake blockchain.

\end{document}
